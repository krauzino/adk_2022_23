\documentclass[oneside,12pt,a4paper]{book}
\usepackage[utf8]{inputenc}
\labelformat{section}{\thechapter.#1}
\usepackage{listings}
\usepackage{FEE}
\pagestyle{CTU_FEE_pagestyle}

%\usepackage{showframe}
\input{acronyms}


\coursename{Algoritmy digitální kartografie a GIS}


\subject{Úloha č. 4:}
\title{Energetické spliny}

\author{Bc. Taťána Bláhová, Bc. Tomáš Krauz, Bc. Adéla Kučerová}


\date{\today} % keep commented to add current date, empty for no date, input version of file (preferred)
\lstset{ 
showstringspaces=false, 
numbers=left, 
numberstyle=\ttfamily\small, 
stepnumber=1, 
firstnumber=last, 
breaklines=T} 




\begin{document}
% \let\cleardoublepage\clearpage
\maketitle

    \vspace{-2cm}
    \vfill
    \begingroup
    
    \tableofcontents
    % \printacronyms
    \vspace{-0.5cm}
    \printglossary[type=\acronymtype,title=\Large Acronyms]
    \endgroup
    \vspace{-1cm}
% \chapter{Example}
% \input{example.tex}


\clearpage
\chapter{Zadání}

\begin{figure}[ht!]
    \centering
    \includegraphics[width=17cm]{fig/u4_zadani.PNG}
    \caption{Zadání úlohy}
    \label{fig:Zadání úlohy}
\end{figure}



%\chapter{Bonusové úlohy} 
%\begin{enumerate}
%    \item \emph{Partial modification: 1 prvek, 2 prvky. 20b}
    %\item \emph{Partial modification: 2 prvky a překážka. +10b} % nemame
   % \item \emph{Partial modification: 3 prvky.    +15b} % nemame
    %\item \emph{Variabilní počet iterací, polohová chyba menší než grafická přednost mapy.     +10b} % nemame
%\end{enumerate}





\chapter{Popis a rozbor problému energetických splinů} 
Úloha je založena na využití generalizačního operátoru Partial modification pro realizování odsunu či částečnou změnu tvaru jednoho prvku vůči pevné bariéře, nebo dvou prvků vzájemně. Tímto odsunem má dojít k odstranění možného grafického konfliktu. Je zde počítáno s jistou minimální vzdáleností prvků. Tato vzdálenost je volena v závislosti na měřítkovém čísle mapy (např. 1 mm v mapě). Implementace je provedena metodou energetických splinů.\par

Energetický spline je používaný v oblasti PC grafiky, kde pro potřeby kartografické generalizace došlo k úpravě definice splinu. Operátor částečné modifikace (partial modification) je tak reprezentován parametrickou křivkou:
\begin{equation}
    d(s)=(s(x)-x_{(0)}(s), y(s)-y_{(0)}(s)),
\end{equation}
kde x(0), y(0) představují souřadnice vstupního prvku a souřadnice x,y jsou souřadnicemi výstupního prvku. Pro parametr s platí: $s \in <0,1>$.\par
Příslušný energetický model pro křivku  \emph{L} a její délku \emph{l} má tvar
\begin{equation}
    E(d)=\int_{l}E_i(s)\mathrm{d}s + \int_{l}E_e(s)\mathrm{d}s,
\end{equation}
a zahrnuje vnitřní energii splinu $E_i$ (internal energy) a vnější energii $E_e$ (external energy), které
ovlivňují tvar splinu. Cílem metody je nalézt jednu z mnoha deformací, která minimalizuje celkovou energii splinu. Výsledný spline musí zohledňovat vnitřní energii (tj. mechanické vlastnosti materiálu), tak i působení vnějších sil. Vnitřní spline zajišťuje ,aby  nebyl modifikován vnějšími silami.

Vnitřní energie splinu je definována vztahem \cite{bader2001energy} a ovlivňuje průběh splinu a jeho tvar.

\begin{equation}
    E_i(s) = \frac{1}{2} \left( \alpha(s)\| d(s)\|^2 + \beta(s) \| \frac{\partial{d}(s)}{\partial{s}}\|^2 + \gamma(s) \| \frac{\partial^2d(s)}{\partial s^2}\|^2 \right),
\end{equation}

První člen měří vzdálenost mezi splinem a původním elementem, druhý elasticitu splinu a poslední tuhost splinu. 
Vliv těchto faktorů je modelován s využitím trojice parametrů 
$\alpha(s),\beta(s),\gamma(s)\in\mathbb{R^+} $. 
Spline tedy může více či méně sledovat původní prvek a kopírovat jeho tvar \cite{kass1988snakes}, \cite{burghardt1997cartographic}.\par

Vnější energie řídí deformaci způsobenou vnějšími silami.  Energetická funkce popisující silový model má mnoho podob. Z matematického pohledu by měla být spojitá v bodě, diferencovatelná a mít jednoduchý průběh bez oscilací. Čím větší jsou funkční hodnoty, tím silnější je jejich vliv na deformaci tvaru. Z kartografické generalizace existuje způsob, jak navrhnout energetickou funkci, jejichž cílem je generalizační operace partial displacement, při kterém se snažíme nepřiblížit k jinému prvku na vzdálenost menší než \textbf{\emph{d}}.

Minimalizace celkové energie, 
\begin{equation}
    E(d(s))=\int_l F(s,d(s),d'(s),d''(s))ds,
\end{equation}
vede k využití Eulerovy-Lagrangeovy rovnice

\begin{equation}
     \frac{\partial F}{\partial y} - \frac{\partial }{\partial x}\frac{\partial F}{\partial y'} +\frac{\partial^2 }{\partial x^2}\frac{\partial F}{\partial y''} - \cdots + (-1)^n \frac{\partial^n }{\partial x^n}\frac{\partial F}{\partial y^n} = 0
\end{equation}

Aplikací na E(d) obdržíme 
\begin{equation}
    \frac{\partial F}{\partial y}= \alpha(s)d(s) + \bigtriangledown E_e (x(s),y(s)),
\end{equation}

\begin{equation}
    \frac{\partial }{\partial x}\frac{\partial F}{\partial y'} = \frac{\partial }{\partial s}
    \left( \beta(s)\frac{\partial d(s)}{\partial s} \right) =  \beta(s)\frac{\partial d^2(s)}{\partial s^2},
\end{equation}

\begin{equation}
    \frac{\partial ^2}{\partial x^2}\frac{\partial F}{\partial y''} = \frac{\partial^2 }{\partial x^2}
    \left( \gamma(s)\frac{\partial^2 d(s)}{\partial s^2} \right) =  \gamma(s)\frac{\partial d^4(s)}{\partial s^4},
\end{equation}
optimální řešení má pak tvar
\begin{equation}
    \alpha(s)d(s) + \beta(s)\frac{\partial d^2(s)}{\partial s^2} -\gamma(s)\frac{\partial d^4(s)}{\partial s^4} + \bigtriangledown E_e (x(s),y(s)) = 0,
\end{equation}

Rozepsáním po složkách dostaneme soustavu dvou parciálních rovnic. Ale počítáme s tím, že hodnoty $\alpha(s), \beta(s), \gamma(s)$ jsou hodnoty konstantní při ekvidistantním kroku \emph{h}, tak je možno získat Eulerovy rovnice v tomto tvaru

\begin{equation}
\alpha x + \beta\frac{\partial^2x}{\partial s^2} - \gamma\frac{\partial^4x}{\partial s^4} + \frac{\partial}{\partial x} E_e (x(s),y(s)) = 0
\end{equation}

\begin{equation}
\alpha y + \beta\frac{\partial^2y}{\partial s^2} - \gamma\frac{\partial^4y}{\partial s^4} + \frac{\partial}{\partial y} E_e (x(s),y(s)) = 0
\end{equation}

Ale pokud je spline vzorkován konstantním krokem \emph{h}, můžeme použít diskrétní aproximaci. Proto je možné parciální derivace nahradit centrálními derivacemi.

Pokud je vložíme do Eulerových rovnic, získáme soustavu lineárních rovnic

\begin{equation}
\alpha x_i+\beta(x_{i-1}-2x_i + x_{i+1})+ \beta(x_{i-1}-2x_i + x_{i+1})+\gamma(x_{i-2}-4x_{i-1} + 
6x_i-4x_{i+1}+x_{i+2})+E_{e,x}
\end{equation}
\begin{equation}
\alpha y_i+\beta(y_{i-1}-2y_i + y_{i+1})+ \beta(y_{i-1}-2y_i + y_{i+1})+\gamma(y_{i-2}-4y_{i-1} + 
6y_i-4y_{i+1}+y_{i+2})+E_{e,y}
\end{equation}

Hodnoty $E_{e,x},E_{e,y}$ představují parciální derivace vnější energie podle proměnných $ x_i.y_i$. Z toho vyplývá, že spline musí být vzorkován nejméně pěti body $ p_{i-2},p_{i-1},p_i,p_{i+1},p_{i+2}$ 
a tak příslušná maticová reprezentace má tvar 

\begin{equation}
A\bigtriangleup X + E_{e,x} = 0,
\end{equation}
\begin{equation}
A\bigtriangleup Y + E_{e,y} = 0,
\end{equation}
kde A je pentadiagonální matice

\begin{equation}
\begin{bmatrix} 
a & b & c & 0& \cdots & 0 & 0 & 0 & 0  \\
b & a & b & c& \cdots & 0 & 0 & 0 & 0  \\
c & b & a & b &\cdots & 0 & 0 & 0 & 0  \\
0 & c & b & a& \cdots & 0 & 0 & 0 & 0  \\
\vdots & \vdots &\vdots &\vdots & \ddots & \vdots &\vdots &\vdots &\vdots \\ 
0 & 0 & 0 & 0 &  \cdots & a & b & c & 0 \\
0 & 0 & 0 & 0 &  \cdots &b & a & b & c \\
0 & 0 & 0 & 0 &  \cdots &c & b & a & b \\
0 & 0 & 0 & 0 & \cdots &0 & c & b & a \\
\end{bmatrix},
\end{equation}
jejiž prvky vypočítáme z 

\begin{equation}
a = \alpha + \frac{2\beta}{h^2}+\frac{6\gamma}{h^4}, \quad b = -\frac{\beta}{h^2}-\frac{4\gamma}{h^4}, \quad 
c = \frac{\gamma}{h^4}
\end{equation}
Protože je matice A singulární, tak hodnoty  $\bigtriangleup X_{(i)},\bigtriangleup Y_{(i)}$ nemůžeme určit přímo a je potřeba je vypočítat pomocí iterace

\begin{equation}
\bigtriangleup X_{(i)} = (A+\lambda I)^{-1}(\lambda \bigtriangleup X_{i-1}- E_{e,x}),
\end{equation}
\begin{equation}
\bigtriangleup Y_{(i)} = (A+\lambda I)^{-1}(\lambda \bigtriangleup Y_{i-1}- E_{e,y}),
\end{equation}
kde 
\begin{equation}
\bigtriangleup X_{(i)} = X_{(i)} - X_{(0)}
\end{equation}
\begin{equation}
\bigtriangleup Y_{(i)} = Y_{(i)} - Y_{(0)}
\end{equation}

jsou představeny jako souřadnicové rozdíly vrcholů splinu v jejich $i$-té iteraci a počáteční aproximací představované lomenou čarou. Pro $i = 0$, platí

\begin{equation}
\bigtriangleup X_{(0)} = Y_{(0)} = 0
\end{equation}

Parametr $\lambda$ ovlivňuje rychlost konvergence iteračního procesu.

\section{Operace Partial Displacement}
Jedná se o generalizační operaci, která provádí komplexní korekci tvaru geometrické polohy a generalizačního prvku. Zahrnuje posun a změnu tvaru častí prvků, které se přiblíží k jinému prvku na určitou mez danou hodnotou \textbf{\emph{\underline{d}}}. Tato generalizace se používá u prvků, které se nacházejí blízko sebe a může dojít k vzájemnému překrytí. A díky tomu existují základní generalizační schémata. V úloze byla použita tato 2 generalizační schémata.

\begin{enumerate}
    \item  Energetické spliny - 1 prvek, 1 bariéra
    \item   Energetické spliny - 2 prvky, žádná bariéra   
\end{enumerate}

Máme energetickou funkci, 

\begin{equation}
 E_e(x,y) =
  \begin{cases}
    c(1-\frac{d}{\underline{d}}),       & \quad d<\underline{d},\\
    0,  & \quad jinak,
  \end{cases}
\end{equation}
která je navržena tak, aby zabránila přiblížení dvou prvků ve vzdálenosti menší než 
$\textbf{\underline{d}}$. Můžeme $\textbf{\underline{d}}$ chápat jako minimální vzdálenost 
při které nedojde ke kartografickému slití v měřítku digitalizace. Vzdálenost je měřena mezi vrcholy 
$p_i $
jednoho prvku a liniovým segmentem druhého prvku. 
Konstanta $c,c\in \mathbb{R^+}$ 
ovlivňuje hodnotu 
gradientu a reguluje spád funkce. Iterativní řešení diskretizované varianty splinu využívá parciální 
derivaci $E_e(x,y)$ dle $x,y$. Pokud $d < \underline{d}$ funkci přepíšeme do tvaru 

\begin{equation}
E_e(x,y) = c \left(1-\sqrt{\frac{(x-x_n)^2 + (y-y_n)^2}{\underline{d}}}\right),
\end{equation}
kde $ q_n = [x_n,y_n]$ je nejbližší vrchol k vrcholu $p = [x,y]$, parciální derivace mají tvar

\begin{equation}
 \frac{\partial E_e(x,y)}{\partial x} = -c\frac{x-x_n}{d\underline{d}}
\end{equation}
\begin{equation}
\frac{\partial E_e(x,y)}{\partial y} = -c\frac{y-y_n}{d\underline{d}}
\end{equation}
Pro $d \geq \underline{d}$, parciální derivace mají tvar
\begin{equation}
 \frac{\partial E_e(x,y)}{\partial x} = \frac{\partial E_e(x,y)}{\partial y} = 0
\end{equation}
Změna polohy vrcholů splinu, ovlivňující tvar splinu, probíhá pouze ve směru $ p \rightarrow q_n$. Výsledné posuny vycházejí z požadavku vyrovnaného stavu mezi vnitřní a vnější energií.


\section{Energetické spliny - 1 prvek, 1 bariéra}
V tomto případě je do aplikace načtena právě jedna bariéra a jeden element, který je bariérou ovlivněn. Při spuštění funkce dojde k modifikaci pouze elementu, který je posunut. Bariéra je považována za pevný prvek, kterým pohnuto nebylo. Generalizovaný tvar je představován polynií $L = \{p_1,...,p_n\}$ s \emph{n} vrcholy $p_i = [x_i,y_i]$, bariéra je polynie $B = [q_1,...,q_m]$ s \emph{m} vrcholy $q_j = [x_j,y_j]$.

Energetická funkce
\begin{equation}
 E_e(x_i,y_i) =
  \begin{cases}
    c(1-\frac{d_i}{\underline{d}}),       & \quad d_i<\underline{d},\\
    0,  & \quad jinak,
  \end{cases}
\end{equation}

Vzdálenost $d_i$ je měřena mezi vrcholem $p_i \in L$ a nejbližším bodem $q_n \in B$. 

\begin{equation}
d_i = \sqrt{(x_i-x_n)^2 + (y_i-y_n)^2}
\end{equation}

Pro $d_i < \underline{d}$, parciální derivace $E_e(x_i,y_i)$ mají tvar

\begin{equation}
 \frac{\partial E_e(x_i,y_i)}{\partial x_i} = -c\frac{x_i-x_n}{d_i\underline{d}}
\end{equation}
\begin{equation}
\frac{\partial E_e(x_i,y_i)}{\partial y_i} = -c\frac{y_i-y_n}{d_i\underline{d}}
\end{equation}
jinak
\begin{equation}
 \frac{\partial E_e(x_i,y_i)}{\partial x_i} = \frac{\partial E_e(x_i,y_i)}{\partial y_i} = 0.
\end{equation}

V rámci celého předzpracování je provedena konverze polylinie na na maticovou podobu. Generalizovaný prvek $L$ je popsán maticemi $X(n,1)$ a $Y(n,1)$, bariéra $B$ maticemi$ X'(m,1)$ a $Y'(n,1)$. Nejdříve se vypočítá krok $h$ ze souřadnicových rozdílů.
\begin{equation}
\delta X = x_{i+1}-x_i,\quad \delta Y = y_{i+1}-y_i,
\end{equation}
 a vzdálenost mezi vrcholy generalizované polylinie
 \begin{equation}
H = \sqrt{ \| \delta X\|^2 +\| \delta Y\|^2}
\end{equation}
Výpočet normy můžeme realizovat s využitím násobení po složkách
Výsledný krok $h$ určíme jako střední hodnotu $H$.

Pro zadané hodnoty $\alpha ,\beta ,\gamma $ a krok $h$ 
určíme hodnoty koeficientů $a, b, c$ a naplní se matice $A$, 
která je během iteračního procesu konstantní. Pokud 
$B = A+ \lambda I$, můžeme vypočítat inverzi $B^{-1}$.
Pro iterační proces nejdříve musíme inicializovat hodnoty matic $X_{(0)} = X$ a $Y_{(0)} = Y$. Pro předem zadaný počet iterací vytvoříme matice  $E_x(n,1)$ a $E_x(n,1)$ a vypočítáme hodnoty prvků. Pro každý bod $p_i \in L, p_i = [x_i,y_i]$ a nalezneme nejbližší bod $q_n \in B, q_n = [x_n,y_n],$ a vypočteme hodnoty parciálních derivací vnější energie  $\frac{\partial E_e(x_i,y_i)}{\partial x_i}, 
\frac{\partial E_e(x_i,y_i)}{\partial y_i}  $
v tomto bodě. Určíme hodnoty posunů $\bigtriangleup X_{(i)}, \bigtriangleup Y_{(i)}$

 \begin{equation}
\bigtriangleup X_{(i)} = B^{-1}(\lambda \bigtriangleup X_{(i-1)} E_{e,x}), 
\quad 
\bigtriangleup Y_{(i)} = B^{-1}(\lambda \bigtriangleup Y_{(i-1)}E_{e,y})
\end{equation}
a vypočítáme souřadnice vrcholů splinu
 \begin{equation}
 X_{(i)} = X + \bigtriangleup X_{(i)},
\quad 
 Y_{(i)} = Y + \bigtriangleup Y_{(i)}
\end{equation}
Inkrementuje se index $i = i+1$. Pokud $i$ < maximální počet iterací, jdi na bod a) jinak ukončí iterační proces. Poté je provedena konverze z maticové reprezentace na spojový seznam vrcholů reprezentující generalizovanou polylinii. 

\section{Energetické spliny - 2 prvky, žádná bariéra}
V tomto případě jsou do aplikace načteny dva elementy, které se vzájemně ovlivní a posunou se vůči sobě. Ani jeden z prvků není brán za pevný. Prvník prvek je představován polynií $L = \{p_1,...,p_n\}$ s \emph{n} vrcholy $p_i = [x_i,y_i]$, druhý prvek je polynie $L' = [q_1,...,q_m]$ s \emph{m} vrcholy $q_j = [x_j,y_j]$. Energetická funkce zohledňuje vzájemný vliv obou prvků. Energetická funkce pro polylinii $L'$ má tvar
\begin{equation}
 E_e(x_i,y_i) =
  \begin{cases}
    c(1-\frac{d_j}{\underline{d}}),       & \quad d_j<\underline{d},\\
    0,  & \quad jinak,
  \end{cases}
\end{equation}
kde $d_j$ představuje vzdálenost mezi vrcholem $q_j \in L'$ a nejbližším bodem $p_n \in L$ 
\begin{equation}
d_j = \sqrt{(x_j-x_n)^2 + (y_j-y_n)^2}
\end{equation}

\begin{equation}
 \frac{\partial E_e(x_j,y_j)}{\partial x_j} = -c\frac{x_j-x_n}{d_j\underline{d}}
\end{equation}
\begin{equation}
\frac{\partial E_e(x_j,y_j)}{\partial y_j} = -c\frac{y_j-y_n}{d_j\underline{d}}
\end{equation}
jinak
\begin{equation}
 \frac{\partial E_e(x_j,y_j)}{\partial x_j} = \frac{\partial E_e(x_j,y_j)}{\partial y_j} = 0.
\end{equation}

Vrcholy obou polylinií, pro které platí $d_i < d$ nebo $d_j < \underline{d}$, se od sebe vzájemně posunují ve směrech 
$p_i \rightarrow q_n$ a $q_j \rightarrow p_n$.

Výpočty parciální derivací energetické funkce je nutné mít tak, aby na změny vrcholu první polylinie mohla reagovat i polylinie druhá. Takže každá polylinie bude mít vlastní matici A. Generalizovaný prvek $L$ je popsán maticemi $X(n,1)$ a $Y(n,1)$, generalizovaný prvek $L'$ je popsán maticemi $X'(m,1)$ a $Y'(n,1)$
Vypočte se krok $h$ ze souřadnicových rozdílů.
\begin{equation}
\delta X = x_{i+1}-x_i,\quad \delta Y = y_{i+1}-y_i, \quad \delta X' = x_{j+1}-x_{ij},\quad \delta Y' = y_{j+1}-y_{ij},
\end{equation}

a vzdálenosti mezi vrcholy obou generalizovaných polylinií
 \begin{equation}
H = \sqrt{ \| \delta X\|^2 +\| \delta Y\|^2}, \quad H' = \sqrt{ \| \delta X'\|^2 +\| \delta Y'\|^2}
\end{equation}

Výsledné kroky $h, h'$ určíme jako střední hodnoty $H,H'$.

Pro zadané hodnoty $\alpha ,\beta ,\gamma $ a krok $h, h'$ 
určíme hodnoty koeficientů $a, b, c$ a naplní se matice $A, A'$, 
které jsou během iteračního procesu konstantní. Pokud 
$B = A+ \lambda I$, $B' = A'+ \lambda I$, můžeme vypočítat inverzi $B^-1$.
Pro iterační proces nejdříve musíme inicializovat hodnoty matic 
$X_{(0)} = X$ a $Y_{(0)} = Y$
\begin{equation}
X_{(0)} = X \quad Y_{(0)} = Y \quad X'_{(0)} = X' \quad Y'_{(0)} = Y'
\end{equation}

Pro předem zadaný počet iterací vytvoříme matice  $E_x(n,1)$, $E_x(n,1)$, $E'_x(m,1)$ a $E'_x(m,1)$ a vypočítají se hodnoty prvků.
Pro každý bod $p_i \in L, p_i = [x_i,y_i]$ a nalezneme nejbližší bod $q_n \in B, q_n = [x_n,y_n],$ a vypočteme hodnoty parciálních derivací vnější energie  $\frac{\partial E_e(x_i,y_i)}{\partial x_i}, 
\frac{\partial E_e(x_i,y_i)}{\partial y_i}  $
v tomto bodě.
Pro každý bod $q_j \in L, q_j = [x_j,y_j]$ a nalezneme nejbližší bod $p_n \in B, q_n = [x_n,y_n],$ a vypočteme hodnoty parciálních derivací vnější energie  $\frac{\partial E_e(x_j,y_j)}{\partial x_j}, 
\frac{\partial E_e(x_j,y_j)}{\partial y_j}  $
v tomto bodě. Určíme hodnoty posunů $\bigtriangleup X_{(i)}, \bigtriangleup Y_{(i)}$
 \begin{equation}
\bigtriangleup X_{(i)} = B^{-1}(\lambda \bigtriangleup X_{(i-1)} E_{e,x}), 
\quad 
\bigtriangleup Y_{(i)} = B^{-1}(\lambda \bigtriangleup Y_{(i-1)}E_{e,y})
\end{equation}
A určíme hodnoty posuny $\bigtriangleup X'_{(i)}, \bigtriangleup Y'_{(i)}$
 \begin{equation}
\bigtriangleup X'_{(i)} = B'^{-1}(\lambda \bigtriangleup X'_{(i-1)} E'_{e,x}), 
\quad 
\bigtriangleup Y'_{(i)} = B'^{-1}(\lambda \bigtriangleup Y'_{(i-1)}E'_{e,y})
\end{equation}

a vypočítáme souřadnice vrcholů splinu.
 \begin{equation}
 X_{(i)} = X + \bigtriangleup X_{(i)},
\quad 
 Y_{(i)} = Y + \bigtriangleup Y_{(i)}
 \quad 
 X'_{(i)} = X' + \bigtriangleup X'_{(i)},
\quad 
 Y'_{(i)} = Y' + \bigtriangleup Y'_{(i)}
\end{equation}
Inkrementuje se index $i = i+1$. Pokud je $i$ menší než maximální počet iterací, jdi na bod a), jinak ukonči iterační proces. Poté je provedena konverze z maticové reprezentace na spojový seznam vrcholů reprezentující generalizovanou polylinii. 

Principem energetických splinů je realizace odsunu či změna tvaru prvků vůči jinému za účelem odstranění grafického konfliktu. K výpočtu je brána v potaz minimální vzdálenost dotčených prvků, která má být mezi objekty.\par
Do výpočtu vstupuje vnitřní a vnější energie splinu, které ovlivňují tvar splinu. Jak je uvedeno výše, cílem metody je nalezení minimální celkové energie splinu. Pokud je spline vzorkován s konstantním krokem $h$, lze použít také jeho diskrétní aproximaci, která je pro praktické výpočty vhodnější. Parciální derivace tak lze nahradit centrálními diferencemi.\par

\chapter{Funkce aplikace}
Výsledkem jsou jedna nebo více posunutých křivek v závislosti na vybrané metodě. Vstupními daty jsou existující kartografická data (liniové prvky). Využitím generalizačního operátu Partial Modification dojde k posouzení jejích vzájemného umístění. V závislosti na vzdálenosti lomových bodů a typu prvku (bariéra, prvek) je buď zachován stávající stav, pokud jsou od sebe prvky dostatečně vzdálené, nebo dojde k posunu prvku elementu, nebo prvků vzájemně, jedná-li se o dva elementy.\par

\pagebreak
\chapter{Vzhled aplikace}

Po spuštění aplikace se zobrazí následující okno, kde je možné provést tyto akce - načíst CSV soubor s prvkem a bariérou (File), změnit aktuálně ručně kreslený prvek (Input), zvolit metodu pro odsunutí (Displace) a vyčistit Canvas (Clear). \par

\begin{figure}[ht!]
    \centering
    \includegraphics[width=18 cm]{fig/vzhled_aplikace.PNG}
    \caption{Vzhled aplikace}
    \label{fig:Zadání úlohy}
\end{figure}

\pagebreak
\chapter{Vstupní data} 
Aplikace očekává na vstupu CSV soubor se souřadnicemi bodů příslušného prvku v souřadnicovém systému S-JTSK. \par
\begin{figure}[ht!]
    \centering
    \includegraphics[width=8.5 cm]{fig/vstup_souradnice.PNG}
    \caption{Formát vstupního souboru}
    \label{fig:Zadání úlohy}
\end{figure}
Pomocí tlačítka File je možné postupně načíst soubor s prvkem a bariérou, přičemž nezáleží na pořadí, v jakém uživatel soubory načte.
\begin{figure}[ht!]
    \centering
    \includegraphics[width=7 cm]{fig/open_file.PNG}
    \caption{Načtení vstupních souborů}
    \label{fig:Zadání úlohy}
\end{figure}


\chapter{Výstupní data} 
Po kliknutí na zvolenou metodu aplikace provede modifikaci buďto jednoho nebo dvou prvků (resp. prvku i bariéry). V případě zvolení metody \emph{Displace 2 elements} se soubor (linie) načtený jako bariéra chová jako prvek a je také posunut. \par
\begin{figure}[ht!]
    \centering
    \includegraphics[width=18 cm]{fig/displace1.PNG}
    \caption{Modifikace 1 prvku (element) - nová poloha prvku vykreslena zeleně}
    \label{fig:csv}
\end{figure}
\begin{figure}[ht!]
    \centering
    \includegraphics[width=18 cm]{fig/displace2.PNG}
    \caption{Modifikace 2 prvků (element i bariéra) - nová poloha prvků vykreslena zeleně a žlutě}
    \label{fig:csv}
\end{figure}

\pagebreak
\chapter{Problematické situace a jejich rozbor}
Při spuštění aplikace nefunguje obyčejné vytvoření linií pomocí ručního klikání a funkce je možné spustit po nahrání externího souboru. Tento problém nebyl prakticky vyřešen. Teoretickým řešením by zřejmě byla duplikace a úprava kódu pro samostatné fungování pro nahrání souboru a ručního klikání společně s toggle tlačítkem, kde by došlo k přepínání mezi těmito dvěma kódy.\par

%Na vstupu během nahrávání je očekáván vždy jeden prvek pro element či bariéru, ačkoliv je očekáván vstup multilinie. Tento problém nebyl vyřešen a nebyla nalezena správná alternativa. \par

Aplikace spadne při pokusu o zavření dialogového okna "Open file with points", důvodem je neošetření varianty, kdy nebyl zvolen žádný soubor.\par

Do aplikace je možnost opakovaného nahrání souboru pro element či bariéru. Tento problém nebyl ošetřen, jelikož se předpokládá, že uživatel nahraje pouze nutné soubory pro element a bariéru, a následně spustí funkci Displace.\par

\pagebreak

\chapter{Dokumentace} 
\section{Třída Algorithms}
    \begin{itemize}
    \item $double \;getEuclDistance(double\; x1,\; double \; y1,\; double \; x2,\; double\ ; y2)$
        \begin{itemize}
        \item  Euklidovská vzdálenost dvou bodů
        \end{itemize}

    \item $std\;:\;:tuple<double,\; double,\; double> getPointLineDistance(double\; x,\; double\; y,\; double\; x1,\; \\double\; y1,\; double\; x2,\; double\; y2)$
        \begin{itemize}
        \item  uložení vzdáleností bodu a linie
        \end{itemize}

    \item $std\;:\;:tuple<double,\; double,\; double> getPointLineSegmentDistance(double\; x,\; double\; y,\; double\; x1,\; \\double\; y1,\; double\; x2,\; double\; y2)$
        \begin{itemize}
        \item  uložení vzdáleností bodu a segmentu linie
        \end{itemize}
        
    \item $std\;:\;:tuple<int,\; double,\; double,\; double> getNearestLineSegmentPoint(double\; xq,\; double\; yq,\; \\Matrix\; \&X,\; Matrix\; \&Y)$
        \begin{itemize}
        \item  získání nejbližšího segmentu linie
        \end{itemize}

    \item $Matrix\; createA(double\; alfa,\; double\; beta,\; double\; gamma,\; double\; h,\; int\; n)$
        \begin{itemize}
            \item  vytvoření matice A
        \end{itemize}

    \item $std::vector<QPointF>\; minEnergySpline1Element1Barrier(std::vector<QPointF>\; element,\;std::vector<QPointF>\; barrier,\; double\; alfa,\; double\; beta,\; double\; gamma,\; \\double\; lambda,\; double\; dmin,\; int\; iter)$
        \begin{itemize}
            \item minimální energetické spliny mezi prvkem a bariérou
        \end{itemize}

    \item $std::tuple<std::vector<QPointF>,\; std::vector<QPointF>>\; \\minEnergySpline2Elements(std::vector<QPointF>\; element1,std::vector<QPointF>\; element2,\; double\; alfa,\; double\; beta,\; double\; gamma,\; double\; lambda,\; double\; dmin,\; int\; iter)$
        \begin{itemize}
            \item  minimální energetické spliny mezi 2 elementy
        \end{itemize}

    \item $double\; getEx(double\; x,\; double\; y,\; double\; xn,\; double\; yn,\; double\; dmin)$
        \begin{itemize}
            \item  získá souřadnice Ex
        \end{itemize}

    \item $double\; getEy(double\; x,\; double\; y,\; double\; xn,\; double\; yn,\; double\; dmin)$
        \begin{itemize}
            \item  získá souřadnice Ey 
        \end{itemize}

    \item $static std::vector<QPointF>\; transformPoints(std::vector<QPointF>\; \&points,\; \\double\; \&trans\_x,\; double\; \&trans\_y,\; double\; \&scale,\; int\; \&offset\_x,\; int\; \&offset\_y)$
        \begin{itemize}
            \item  transformuje souřadnice bodů pro tvorbu min-max boxu
        \end{itemize}      
    \end{itemize}

   \section{Třída Draw} 
    \begin{itemize}
    
    \item $void\; mousePressEvent(QMouseEvent *event)$
     \begin{itemize}
     \item  vrátí souřadnice kurzoru po kliknutí na Canvas
     \end{itemize}
     
    \item $void\; paintEvent(QPaintEvent *event)$
    \begin{itemize}
    \item  vykreslí body na Canvas
    \end{itemize}
    
    \item $void\; switchDrawnElement()\{draw\_element\;=!\;draw\_element;\}$
    \begin{itemize}
    \item  změna typu kreslení bariéra/prvek
    \end{itemize}

    \item $std::vector<QPointF>getElement()\{return \; pointsElement;\}$
    \begin{itemize}
    \item vrátí prvek
    \end{itemize}
    
    \item $std::vector<QPointF>getBarrier()\{return\; pointsBarrier;\}$
    \begin{itemize}
    \item  vrátí bariéru
    \end{itemize}

    \item $std::vector<QPointF>getDisplaced()\{return\; displaced;\}$
    \begin{itemize}
    \item  posun
    \end{itemize}

    \item $void\; setElement(std::vector<QPointF> \&element\_)\{element=element\_;\}$
    \begin{itemize}
    \item  nastavení prvku
    \end{itemize}

    \item $void\; setBarrier(std::vector<QPointF> \&barrier\_)\{barrier=barrier\_;\}$
    \begin{itemize}
    \item  nastavení bariéry
    \end{itemize}

    \item $void\; setDisplaced(std::vector<QPointF> \&displaced\_)\{displaced=displaced\_;\}$
    \begin{itemize}
    \item  nastavení posunů 1 prvku
    \end{itemize}

    \item $void\; setDisplaced2(std::vector<QPointF> \&displaced2\_)\{displaced2 = displaced2\_;\}$
    \begin{itemize}
    \item  nastavení posunů 2 prvků
    \end{itemize}
    
    \item $void\; drawCSVPointsElement(std::vector<QPointF> \&pointsElement);$
    \begin{itemize}
    \item  vykreslí načtené body prvku na Canvas
    \end{itemize}

    \item $void\; drawCSVPointsBarrier(std::vector<QPointF> \&pointsBarrier);$
    \begin{itemize}
    \item  vykreslí načtené body bariéry na Canvas
    \end{itemize}

  \begin{comment} % proc block comment nefunguje?
    %\item $void\; setCSVPointsElement(std::vector<QPointF> \&csv\_points)\\{pointsElement.insert(pointsElement.end(),\; csv\_points.begin(),\; csv\_points.end());}$
    \begin{itemize}
  %  \item  ----------------------
    \end{itemize}

  %  \item $void\; setCSVPointsBarrier(std::vector<QPointF> \&csv\_points)\\{pointsBarrier.insert(pointsBarrier.end(),\; csv\_points.begin(),\; csv\_points.end());}$
    \begin{itemize}
   % \item  ----------------------
    \end{itemize}

   % \item $double\; getScale()\{return\; scale;\}$
    \begin{itemize}
   % \item  -----------------------
    \end{itemize}
    
   % \item $double\; getTransX()\{return\; trans\_x;\}$
    \begin{itemize}
   % \item  -----------------------
    \end{itemize}

    
   % \item $double\; getTransY()\{return\; trans\_y;\}$
    \begin{itemize}
   % \item  -----------------------
    \end{itemize}

 %   \item $int\; getDeltaX()\{return\; offset\_x;\}$
    \begin{itemize}
  %  \item  -----------------------
    \end{itemize}

   % \item $int\; getDeltaY()\{return\; offset\_y;\}$
    \begin{itemize}
  %  \item  -----------------------
    \end{itemize}

  %  \item $void\; setScale(double\; \&scale\_)\{scale=scale\_;\}$
    \begin{itemize}
  %  \item  -----------------------
    \end{itemize}

  %  \item $void\; setTrans(double\; \&trans\_x\_,\; double\; \&trans\_y\_)\{trans\_x=trans\_x\_,\; trans\_y=trans\_y\_;\}$
    \begin{itemize}
  %  \item  -----------------------
    \end{itemize}

   % \item $void\; setOffsets(int\; \&offset\_x\_,\; int\; \&offset\_y\_)\{offset\_x=offset\_x\_,\; offset\_y=offset\_y\_;\}$
  %  \begin{itemize}
   % \item  -----------------------
%    \end{itemize}
    \end{comment}

    \item $void\; clearElement()\{element.clear();\}$
    \begin{itemize}
    \item  vymaže prvek z Canvasu
    \end{itemize}

    \item $void\; clearBarrier()\{barrier.clear();\}$
    \begin{itemize}
    \item  vymaže bariéru z Canvasu
    \end{itemize}

    \item $void\; clearDisplaced()\{displaced.clear();\}$
    \begin{itemize}
    \item  vymaže odsunutý prvek z Canvasu
    \end{itemize}

    \item $void\; clearDisplaced2()\{displaced2.clear();\}$
    \begin{itemize}
    \item  vymaže odsunutou bariéru (druhý prvek) z Canvasu
    \end{itemize}

    \item $void\; clearPointsElement()\{pointsElement.clear();\}$
    \begin{itemize}
    \item  vymaže načtené body prvku z Canvasu
    \end{itemize}

    \item $void\; clearPointsBarrier()\{pointsBarrier.clear();\}$
    \begin{itemize}
    \item  vymaže načtené body bariéry z Canvasu
    \end{itemize}

\end{itemize}

   \section{Třída CSV}
   \begin{itemize}
    \item $static\; std::vector<std::vector<std::string>>\; readCSV(std::string\; \&filename);$
    \begin{itemize}
\item načte vstupní CSV soubor
\end{itemize}
\item $static\; std::vector<QPointF>\; getPoints(std::vector<std::vector<std::string>>\; \&csv\_content,\; double\; \&x\_min,\; double\; \&x\_max,\; double\; \&y\_min,\; double\; \&y\_max);$
\begin{itemize}
\item vrací vektor QPointF z CSV souboru
\end{itemize}
\end{itemize}

   \section{Třída Mainform}
    \begin{itemize}

    \item $void\; on\_actionOpen\_element\_triggered();$
    \begin{itemize}
        \item  akce pro načtení vstupního souboru s elementem
    \end{itemize}
  
    \item $void\; on\_actionOpen\_barrier\_triggered();$
    \begin{itemize}
        \item  akce pro načtení vstupního souboru s bariérou
    \end{itemize}
    
    \item $void\; on\_actionElement\_Barrier\_changed();$
    \begin{itemize}
        \item  akce po stisknutí tlačítka s prvkem a bariérou
    \end{itemize}
    
    \item $void\; on\_actionDisplace\_triggered();$
    \begin{itemize}
        \item  akce po stisknutí tlačítka Displace 1 element/Displace 2 elements
    \end{itemize}

    \item $void\; on\_actionDisplace\_2\_elements\_triggered();$
    \begin{itemize}
        \item  akce po stisknutí tlačítka se 2 prvky
    \end{itemize}
        
    \item $void\; on\_actionClear\_all();$
    \begin{itemize}
        \item  akce pro vyčištění Canvasu
    \end{itemize}

\end{itemize}

\pagebreak
\chapter{Závěr} 
Byla vytvořena aplikace pro realizaci odsunu a částečnou změnu tvaru elementu pomocí generalizačního operátoru Partial Modification, aby se předešlo grafickému konfliktu. Aplikace byla napsána pro stavy prvek a prvek, nebo prvek a bariéra, kdy modifikovaným prvkem je vždy element a bariéra zůstane beze změny.\par



\begingroup
    \pageclear
    \printbibliography[title=Literatura]
\endgroup

% \begingroup
%     \let\clearpage\relax
%     \printbibliography[title=References]
% \endgroup

\end{document}